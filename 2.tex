首先,我们来确定函数 \( f(x) = 2\sin(wx + \varphi) \) 的参数 \( w \) 和 \( \varphi \)。该函数通过点 (0, 1) 和 (1/2, 0)。

1. 当 \( x = 0 \),\( f(0) = 2\sin(\varphi) = 1 \)。即 \( \sin(\varphi) = \frac{1}{2} \),其中 \( -\frac{\pi}{2} < \varphi < \frac{\pi}{2} \),所以 \( \varphi = \frac{\pi}{6} \) 或 \( \varphi = -\frac{5\pi}{6} \)(但后者不在给定的范围内)。

   因此,\( \varphi = \frac{\pi}{6} \)。

2. 当 \( x = \frac{1}{2} \),\( f\left(\frac{1}{2}\right) = 2\sin\left(w\frac{1}{2} + \frac{\pi}{6}\right) = 0 \),由于 \( \sin \) 函数周期为 \( 2\pi \),我们可以找到 \( w \) 的值,以便该表达式等于 \( n\pi \),\( n \) 为整数。

由于 \( f\left(\frac{1}{2}\right) = 0 \),我们知道 \(w\frac{1}{2} + \frac{\pi}{6}\) 必须等于 \( k\pi \),此处 \( k \) 是整数。为了满足 \( w < 0 \) 的条件,我们可以设 \( k = -1 \),因此得到:

\[ w\frac{1}{2} + \frac{\pi}{6} = -\pi \]

解这个方程求 \( w \):

\[ w\frac{1}{2} = -\pi - \frac{\pi}{6} \]

\[ w\frac{1}{2} = -\frac{7\pi}{6} \]

\[ w = -\frac{14\pi}{6} \]

\[ w = -\frac{7\pi}{3} \]

函数 \( f(x) = 2\sin\left(-\frac{7\pi}{3}x + \frac{\pi}{6}\right) \)。

接着考察 \( f(x+5) \) 是否等于 \( f(5-x) \)。

\[ f(x+5) = 2\sin\left(-\frac{7\pi}{3}(x+5) + \frac{\pi}{6}\right) \]

\[ f(5-x) = 2\sin\left(-\frac{7\pi}{3}(5-x) + \frac{\pi}{6}\right) \]

使用正弦函数的奇偶性,\( \sin(-\theta) = -\sin(\theta) \) ,我们可以发现二者不相等,因为 \( x+5 \) 和 \( 5-x \) 是相互关于原点对称的数。

因此,\( f(x+5) \) 不等于 \( f(5-x) \)。
